% Options for packages loaded elsewhere
\PassOptionsToPackage{unicode}{hyperref}
\PassOptionsToPackage{hyphens}{url}
%
\documentclass[
]{article}
\usepackage{amsmath,amssymb}
\usepackage{lmodern}
\usepackage{iftex}
\ifPDFTeX
  \usepackage[T1]{fontenc}
  \usepackage[utf8]{inputenc}
  \usepackage{textcomp} % provide euro and other symbols
\else % if luatex or xetex
  \usepackage{unicode-math}
  \defaultfontfeatures{Scale=MatchLowercase}
  \defaultfontfeatures[\rmfamily]{Ligatures=TeX,Scale=1}
\fi
% Use upquote if available, for straight quotes in verbatim environments
\IfFileExists{upquote.sty}{\usepackage{upquote}}{}
\IfFileExists{microtype.sty}{% use microtype if available
  \usepackage[]{microtype}
  \UseMicrotypeSet[protrusion]{basicmath} % disable protrusion for tt fonts
}{}
\makeatletter
\@ifundefined{KOMAClassName}{% if non-KOMA class
  \IfFileExists{parskip.sty}{%
    \usepackage{parskip}
  }{% else
    \setlength{\parindent}{0pt}
    \setlength{\parskip}{6pt plus 2pt minus 1pt}}
}{% if KOMA class
  \KOMAoptions{parskip=half}}
\makeatother
\usepackage{xcolor}
\IfFileExists{xurl.sty}{\usepackage{xurl}}{} % add URL line breaks if available
\IfFileExists{bookmark.sty}{\usepackage{bookmark}}{\usepackage{hyperref}}
\hypersetup{
  hidelinks,
  pdfcreator={LaTeX via pandoc}}
\urlstyle{same} % disable monospaced font for URLs
\usepackage{color}
\usepackage{fancyvrb}
\newcommand{\VerbBar}{|}
\newcommand{\VERB}{\Verb[commandchars=\\\{\}]}
\DefineVerbatimEnvironment{Highlighting}{Verbatim}{commandchars=\\\{\}}
% Add ',fontsize=\small' for more characters per line
\newenvironment{Shaded}{}{}
\newcommand{\AlertTok}[1]{\textcolor[rgb]{1.00,0.00,0.00}{\textbf{#1}}}
\newcommand{\AnnotationTok}[1]{\textcolor[rgb]{0.38,0.63,0.69}{\textbf{\textit{#1}}}}
\newcommand{\AttributeTok}[1]{\textcolor[rgb]{0.49,0.56,0.16}{#1}}
\newcommand{\BaseNTok}[1]{\textcolor[rgb]{0.25,0.63,0.44}{#1}}
\newcommand{\BuiltInTok}[1]{#1}
\newcommand{\CharTok}[1]{\textcolor[rgb]{0.25,0.44,0.63}{#1}}
\newcommand{\CommentTok}[1]{\textcolor[rgb]{0.38,0.63,0.69}{\textit{#1}}}
\newcommand{\CommentVarTok}[1]{\textcolor[rgb]{0.38,0.63,0.69}{\textbf{\textit{#1}}}}
\newcommand{\ConstantTok}[1]{\textcolor[rgb]{0.53,0.00,0.00}{#1}}
\newcommand{\ControlFlowTok}[1]{\textcolor[rgb]{0.00,0.44,0.13}{\textbf{#1}}}
\newcommand{\DataTypeTok}[1]{\textcolor[rgb]{0.56,0.13,0.00}{#1}}
\newcommand{\DecValTok}[1]{\textcolor[rgb]{0.25,0.63,0.44}{#1}}
\newcommand{\DocumentationTok}[1]{\textcolor[rgb]{0.73,0.13,0.13}{\textit{#1}}}
\newcommand{\ErrorTok}[1]{\textcolor[rgb]{1.00,0.00,0.00}{\textbf{#1}}}
\newcommand{\ExtensionTok}[1]{#1}
\newcommand{\FloatTok}[1]{\textcolor[rgb]{0.25,0.63,0.44}{#1}}
\newcommand{\FunctionTok}[1]{\textcolor[rgb]{0.02,0.16,0.49}{#1}}
\newcommand{\ImportTok}[1]{#1}
\newcommand{\InformationTok}[1]{\textcolor[rgb]{0.38,0.63,0.69}{\textbf{\textit{#1}}}}
\newcommand{\KeywordTok}[1]{\textcolor[rgb]{0.00,0.44,0.13}{\textbf{#1}}}
\newcommand{\NormalTok}[1]{#1}
\newcommand{\OperatorTok}[1]{\textcolor[rgb]{0.40,0.40,0.40}{#1}}
\newcommand{\OtherTok}[1]{\textcolor[rgb]{0.00,0.44,0.13}{#1}}
\newcommand{\PreprocessorTok}[1]{\textcolor[rgb]{0.74,0.48,0.00}{#1}}
\newcommand{\RegionMarkerTok}[1]{#1}
\newcommand{\SpecialCharTok}[1]{\textcolor[rgb]{0.25,0.44,0.63}{#1}}
\newcommand{\SpecialStringTok}[1]{\textcolor[rgb]{0.73,0.40,0.53}{#1}}
\newcommand{\StringTok}[1]{\textcolor[rgb]{0.25,0.44,0.63}{#1}}
\newcommand{\VariableTok}[1]{\textcolor[rgb]{0.10,0.09,0.49}{#1}}
\newcommand{\VerbatimStringTok}[1]{\textcolor[rgb]{0.25,0.44,0.63}{#1}}
\newcommand{\WarningTok}[1]{\textcolor[rgb]{0.38,0.63,0.69}{\textbf{\textit{#1}}}}
\setlength{\emergencystretch}{3em} % prevent overfull lines
\providecommand{\tightlist}{%
  \setlength{\itemsep}{0pt}\setlength{\parskip}{0pt}}
\setcounter{secnumdepth}{-\maxdimen} % remove section numbering
\ifLuaTeX
  \usepackage{selnolig}  % disable illegal ligatures
\fi

\author{}
\date{}

\begin{document}

\# 点分治 \& 动态点分治

\hypertarget{ux6027ux8d28}{%
\subsection{性质}\label{ux6027ux8d28}}

一类将问题分治解决的算法。

一般的流程都是:

\begin{quote}
\begin{itemize}
\item
  找重心。
\item
  依次遍历每个子树:先统计答案。当某点与维护的状态集满足互补关系(即符合条件)就更新答案。当遍历完一次子树后,将子树的状态添加到状态集中。
\item
  遍历子树删除状态集。
\item
  递归子树。
\end{itemize}
\end{quote}

\hypertarget{ux52a8ux6001ux70b9ux5206ux6cbb}{%
\subsection{动态点分治}\label{ux52a8ux6001ux70b9ux5206ux6cbb}}

动态点分治用来解决 \textbf{带点权/边权修改} 的树上路径信息统计问题。

\hypertarget{ux70b9ux5206ux6811}{%
\subsubsection{点分树}\label{ux70b9ux5206ux6811}}

回顾点分治的计算过程。

对于一个结点 \(x\) 来说,其子树中的简单路径包括两种:经过结点 \(x\)
的,由一条或两条从 \(x\) 出发的路径组成的;和不经过结点 \(x\)
的,即已经包含在其所有儿子结点子树中的路径。

对于一个子树中简单路径的计算,我们选择一个分治中心
\(rt\),计算经过该节点的子树中路径的信息,然后对于其每个儿子结点,将删去
\(rt\)
后该点所在连通块作为一个子树,递归计算。选择的分治中心点可以构成一个树形结构,称为
\textbf{点分树}。我们发现,计算点分树中同一层的结点所代表的连通块(即以该结点为分治中心的连通块)的大小总和是
\(O(n)\)
的。这意味着,点分治的时间复杂度是与点分树的深度相关的,若点分树的深度为
\(h\),则点分治的复杂度为 \(O(nh)\)。

可以证明,当我们每次选择连通块的重心作为分治中心的时候,点分树的深度最小,为
\(O(\log n)\) 的。这样,我们就可以在 \(O(n\log n)\)
的时间复杂度内统计树上 \(O(n^2)\) 条路径的信息了。

由于树的形态在动态点分治的过程中不会改变,所以点分树的形态在动态点分治的过程中也不会改变。

下面给出求点分树的参考代码:

\begin{Shaded}
\begin{Highlighting}[]
\DataTypeTok{void}\NormalTok{ calcsiz}\OperatorTok{(}\DataTypeTok{int}\NormalTok{ x}\OperatorTok{,} \DataTypeTok{int}\NormalTok{ f}\OperatorTok{)} \OperatorTok{\{}
\NormalTok{  siz}\OperatorTok{[}\NormalTok{x}\OperatorTok{]} \OperatorTok{=} \DecValTok{1}\OperatorTok{;}
\NormalTok{  maxx}\OperatorTok{[}\NormalTok{x}\OperatorTok{]} \OperatorTok{=} \DecValTok{0}\OperatorTok{;}
  \ControlFlowTok{for} \OperatorTok{(}\DataTypeTok{int}\NormalTok{ j }\OperatorTok{=}\NormalTok{ h}\OperatorTok{[}\NormalTok{x}\OperatorTok{];}\NormalTok{ j}\OperatorTok{;}\NormalTok{ j }\OperatorTok{=}\NormalTok{ nxt}\OperatorTok{[}\NormalTok{j}\OperatorTok{])}
    \ControlFlowTok{if} \OperatorTok{(}\NormalTok{p}\OperatorTok{[}\NormalTok{j}\OperatorTok{]} \OperatorTok{!=}\NormalTok{ f }\OperatorTok{\&\&} \OperatorTok{!}\NormalTok{vis}\OperatorTok{[}\NormalTok{p}\OperatorTok{[}\NormalTok{j}\OperatorTok{]])} \OperatorTok{\{}
\NormalTok{      calcsiz}\OperatorTok{(}\NormalTok{p}\OperatorTok{[}\NormalTok{j}\OperatorTok{],}\NormalTok{ x}\OperatorTok{);}
\NormalTok{      siz}\OperatorTok{[}\NormalTok{x}\OperatorTok{]} \OperatorTok{+=}\NormalTok{ siz}\OperatorTok{[}\NormalTok{p}\OperatorTok{[}\NormalTok{j}\OperatorTok{]];}
\NormalTok{      maxx}\OperatorTok{[}\NormalTok{x}\OperatorTok{]} \OperatorTok{=}\NormalTok{ max}\OperatorTok{(}\NormalTok{maxx}\OperatorTok{[}\NormalTok{x}\OperatorTok{],}\NormalTok{ siz}\OperatorTok{[}\NormalTok{p}\OperatorTok{[}\NormalTok{j}\OperatorTok{]]);}
    \OperatorTok{\}}
\NormalTok{  maxx}\OperatorTok{[}\NormalTok{x}\OperatorTok{]} \OperatorTok{=}
\NormalTok{      max}\OperatorTok{(}\NormalTok{maxx}\OperatorTok{[}\NormalTok{x}\OperatorTok{],}\NormalTok{ sum }\OperatorTok{{-}}\NormalTok{ siz}\OperatorTok{[}\NormalTok{x}\OperatorTok{]);}  \CommentTok{// maxx[x] 表示以 x 为根时的最大子树大小}
  \ControlFlowTok{if} \OperatorTok{(}\NormalTok{maxx}\OperatorTok{[}\NormalTok{x}\OperatorTok{]} \OperatorTok{\textless{}}\NormalTok{ maxx}\OperatorTok{[}\NormalTok{rt}\OperatorTok{])}
\NormalTok{    rt }\OperatorTok{=}\NormalTok{ x}\OperatorTok{;}  \CommentTok{// 这里不能写 \textless{}= ,保证在第二次 calcsiz 时 rt 不改变}
\OperatorTok{\}}
\DataTypeTok{void}\NormalTok{ pre}\OperatorTok{(}\DataTypeTok{int}\NormalTok{ x}\OperatorTok{)} \OperatorTok{\{}
\NormalTok{  vis}\OperatorTok{[}\NormalTok{x}\OperatorTok{]} \OperatorTok{=} \KeywordTok{true}\OperatorTok{;}  \CommentTok{// 表示在之后的过程中不考虑 x 这个点}
  \ControlFlowTok{for} \OperatorTok{(}\DataTypeTok{int}\NormalTok{ j }\OperatorTok{=}\NormalTok{ h}\OperatorTok{[}\NormalTok{x}\OperatorTok{];}\NormalTok{ j}\OperatorTok{;}\NormalTok{ j }\OperatorTok{=}\NormalTok{ nxt}\OperatorTok{[}\NormalTok{j}\OperatorTok{])}
    \ControlFlowTok{if} \OperatorTok{(!}\NormalTok{vis}\OperatorTok{[}\NormalTok{p}\OperatorTok{[}\NormalTok{j}\OperatorTok{]])} \OperatorTok{\{}
\NormalTok{      sum }\OperatorTok{=}\NormalTok{ siz}\OperatorTok{[}\NormalTok{p}\OperatorTok{[}\NormalTok{j}\OperatorTok{]];}
\NormalTok{      rt }\OperatorTok{=} \DecValTok{0}\OperatorTok{;}
\NormalTok{      maxx}\OperatorTok{[}\NormalTok{rt}\OperatorTok{]} \OperatorTok{=}\NormalTok{ inf}\OperatorTok{;}
\NormalTok{      calcsiz}\OperatorTok{(}\NormalTok{p}\OperatorTok{[}\NormalTok{j}\OperatorTok{],} \OperatorTok{{-}}\DecValTok{1}\OperatorTok{);}
\NormalTok{      calcsiz}\OperatorTok{(}\NormalTok{rt}\OperatorTok{,} \OperatorTok{{-}}\DecValTok{1}\OperatorTok{);}  \CommentTok{// 计算两次,第二次求出以 rt 为根时的各子树大小}
\NormalTok{      fa}\OperatorTok{[}\NormalTok{rt}\OperatorTok{]} \OperatorTok{=}\NormalTok{ x}\OperatorTok{;}
\NormalTok{      pre}\OperatorTok{(}\NormalTok{rt}\OperatorTok{);}  \CommentTok{// 记录点分树上的父亲}
    \OperatorTok{\}}
\OperatorTok{\}}
\DataTypeTok{int}\NormalTok{ main}\OperatorTok{()} \OperatorTok{\{}
\NormalTok{  sum }\OperatorTok{=}\NormalTok{ n}\OperatorTok{;}
\NormalTok{  rt }\OperatorTok{=} \DecValTok{0}\OperatorTok{;}
\NormalTok{  maxx}\OperatorTok{[}\NormalTok{rt}\OperatorTok{]} \OperatorTok{=}\NormalTok{ inf}\OperatorTok{;}
\NormalTok{  calcsiz}\OperatorTok{(}\DecValTok{1}\OperatorTok{,} \OperatorTok{{-}}\DecValTok{1}\OperatorTok{);}
\NormalTok{  calcsiz}\OperatorTok{(}\NormalTok{rt}\OperatorTok{,} \OperatorTok{{-}}\DecValTok{1}\OperatorTok{);}
\NormalTok{  pre}\OperatorTok{(}\NormalTok{rt}\OperatorTok{);}
\OperatorTok{\}}
\end{Highlighting}
\end{Shaded}

\hypertarget{ux5b9eux73b0ux4feeux6539}{%
\subsubsection{实现修改}\label{ux5b9eux73b0ux4feeux6539}}

在查询和修改的时候,我们在点分树上暴力跳父亲修改。由于点分树的深度最多是
\(O(\log n)\) 的,所以这样做复杂度能得到保证。

在动态点分治的过程中,需要一个结点到其点分树上的祖先的距离等其他信息,由于一个点最多有
\(O(\log n)\) 个祖先,我们可以在计算点分树时额外计算深度 \(dep[x]\)
或使用
LCA,预处理出这些距离或实现实时查询。\textbf{注意}:一个结点到其点分树上的祖先的距离不一定递增,不能累加!

在动态点分治的过程中,一个结点在其点分树上的祖先结点的信息中可能会被重复计算,这是我们需要消去重复部分的影响。一般的方法是对于一个连通块用两种方式记录:一个是其到分治中心的距离信息,另一个是其到点分树上分治中心父亲的距离信息。这一部分内容将在例题中得到展现。

??? note "
例题\href{https://www.luogu.com.cn/problem/P2056}{「ZJOI2007」捉迷藏}"\\
给定一棵有 \(n\)
个结点的树,初始时所有结点都是黑色的。你需要实现以下两种操作:\\

\begin{Shaded}
\begin{Highlighting}[]
\NormalTok{1. 反转一个结点的颜色(白变黑,黑变白);}
\NormalTok{2.  询问树上两个最远的黑点的距离。}

\NormalTok{    $n\textbackslash{}le 10\^{}5,m\textbackslash{}le 5\textbackslash{}times 10\^{}5$}
\end{Highlighting}
\end{Shaded}

求出点分树,对于每个结点 \(x\) 维护两个 \textbf{可删堆}。\(dist[x]\)
存储结点 \(x\) 代表的连通块中的所有黑点到 \(x\) 的距离信息,\(ch[x]\)
表示结点 \(x\) 在点分树上的所有儿子和它自己中的黑点到 \(x\)
的距离信息,由于本题贪心的求答案方法,且两个来自于同一子树的路径不能成为一条完成的路径,我们只在这个堆中插入其自己的值和其每个子树中的最大值。我们发现,\(ch[x]\)
中最大的两个值(如果没有两个就是所有值)的和就是分治时分支中心为 \(x\)
时经过结点 \(x\) 的最长黑端点路径。我们可以用可删堆 \(ans\)
存储所有结点的答案,这个堆中的最大值就是我们所求的答案。

我们可以根据上面的定义维护 \(dist[x],ch[x],ans\) 这些可删堆。当
\(dist[x]\) 中的值发生变化时,我们也可以在 \(O(\log n)\)
的时间复杂度内维护 \(ch[x],ans\)。

现在我们来看一下,当我们反转一个点的颜色时,\(dist[x]\)
值会发生怎样的改变。当结点原来是黑色时,我们要进行的是删除操作;当结点原来是白色时,我们要进行的是插入操作。

假如我们要反转结点 \(x\) 的颜色。对于其所有祖先 \(u\),我们在
\(dist[u]\) 中插入或删除 \(dist(x,u)\),并同时维护 \(ch[x],ans\)
的值。特别的,我们要在 \(ch[x]\) 中插入或删除值 \(0\)。

参考代码:

\begin{Shaded}
\begin{Highlighting}[]
  \OperatorTok{{-}{-}}\DecValTok{8}\OperatorTok{\textless{}{-}{-}} \StringTok{"docs/graph/code/dynamic{-}tree{-}divide/dynamic{-}tree{-}divide\_1.cpp"}
\end{Highlighting}
\end{Shaded}

???+note " 例题\href{https://www.luogu.com.cn/problem/P6329}{Luogu P6329
【模板】点分树 \textbar{} 震波}"\\
给定一棵有 \(n\) 个结点的树,树上每个结点都有一个权值
\(v[x]\)。实现以下两种操作:\\

\begin{Shaded}
\begin{Highlighting}[]
\NormalTok{1. 询问与结点 $x$ 距离不超过 $y$ 的结点权值和;}
\NormalTok{2. 修改结点 $x$ 的点权为 $y$,即 $v[x]=y$。}
\end{Highlighting}
\end{Shaded}

我们用动态开点权值线段树记录距离信息。

类似于上题的思路,对于每个结点,我们维护线段树 \(dist[x]\),表示分治块
\(x\) 中的所有结点到结点 \(x\)
的距离信息,下标为距离,权值加上点权。线段树 \(ch[x]\) 表示分治块 \(x\)
中所有结点到结点 \(x\) 在分治树上的父亲结点的距离信息。

在本题中,所有查询和修改都需要在点分树上对所有祖先进行修改。

以查询操作为例,如果我们要查询距离结点 \(x\) 不超过 \(y\)
的结点的权值和,我们要先将答案加上线段树 \(dist[x]\) 中下标从 \(0\) 到
\(y\) 的权值和,然后我们遍历 \(x\) 的所有祖先 \(u\),设其低一级祖先为
\(v\),令 \(d=dist(x,u)\),如果我们不进入包含 \(x\) 的子树,即以 \(v\)
为根的子树,那么我们要将答案加上线段树 \(dist[u]\) 中下标从 \(0\) 到
\(y-d\) 的权值和。由于我们重复计算了以 \(v\)
为根的部分,我们要将答案减去线段树 \(ch[v]\) 中下标从 \(0\) 到 \(y-d\)
的权值和。

在进行修改操作时,我们要同时维护 \(dist[x]\) 和 \(ch[x]\)。

参考代码:

\end{document}
